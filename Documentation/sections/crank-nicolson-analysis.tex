\subsection{Crank-Nicolson scheme} \label{s:general-approach:crank-nicolson-analysis}
	Based on formulations given by S. Scott Collis in \cite{bib:introduction} and George Em Karniadakis and Robert M. Kirby II in \cite{bib:mpi} for Crank--Nicolson scheme to solve \gls{advection-equation} can be described as:
	\begin{equation} \label{eq:crank-nicolson}
		\frac{f_i^{n+1} - f_i^n}{\Delta t} +  u \frac{1}{2} \bigg(\frac{f_{i+1}^{n+1} - f_{i-1}^{n+1}}{2\Delta x} + \frac{f_{i+1}^{n} - f_{i-1}^n}{2\Delta x}\bigg)= \mathcal{O}(\Delta t, \Delta x)
	\end{equation}
	Equation \eqref{eq:crank-nicolson} can be transformed in a following fashion to separate knowns and unknowns:
	\begin{equation} \label{eq:crank-nicolson:solution}
		\frac{C}{4} f_{i-1}^n + f_{i}^n - \frac{C}{4} f_{i+1}^n = -\frac{C}{4} f_{i-1}^{n+1} + f_{i}^{n+1} + \frac{C}{4} f_{i+1}^{n+1}
	\end{equation}
	where $C$ is \gls{CFL} number introduced earlier \eqref{eq:cfl}. Assuming following dependencies:
	\begin{equation}
		\begin{split}
			\alpha &= \frac{C}{4}\\
			 k_i &= \frac{C}{4} f_{i-1}^n + f_{i}^n - \frac{C}{4} f_{i+1}^n
		\end{split}
	\end{equation}
	equation \ref{eq:crank-nicolson:solution} drives us to solve following linear system of form $Af^{n+1} = f^n$ by Thomas algorithm:		
	\begin{equation}
		\begin{bmatrix}
			1 		& \alpha &		  &  	 &		&\\
			-\alpha & 1 	 & \alpha &  	 &		&\\ 
					& \ddots & \ddots &  \ddots	 &		&\\
					&		 &-\alpha & 1 	 &\alpha&\\
					& 		 & 		  &-\alpha&	1	&\\					
		\end{bmatrix} 
		\times
		\begin{bmatrix}
			f_1^{n+1} \\
			f_2^{n+1} \\
			\vdots	\\
			f_{N-1}^{n+1}\\
			f_{N}^{n+1}\\
		\end{bmatrix}
		=
		\begin{bmatrix}
			k_0 \\
			k_1 \\
			\vdots	\\
			k_{N-1} \\
			k_{N} \\
		\end{bmatrix}
	\end{equation}		
	Geometric arrangement of a nodal group that relate to the points of interest used by this scheme is shown in Figure \ref{fig:stencil:crank-nicolson}.
	\begin{figure}[!htbp]
		\centering
		\begin{tikzpicture}[scale=1.5]
			\draw[dotted] (-4,0.5) -- (4,0.5);
			\draw[dotted] (-4,-0.5) -- (4,-0.5);
			\node[black] at (-4.5,1) {$n+1$};
			\node[black] at (-4.5,0) {$n$};
			
			\node[black] at (-1,-1) {$i-1$};
			\node[black] at (0,-1) {$i$};
			\node[black] at (1,-1) {$i+1$};
			
			\stencilpt[fill=blue]{0,0}{ij}{};
			\stencilpt[fill=blue]{-1,0}{i-1j}{};			
			\stencilpt[fill=blue]{1,0}{i+1j}{};
			
			\stencilpt[fill=green]{0,1}{ij+1}{};
			\stencilpt[fill=blue]{-1,1}{i-1j+1}{};			
			\stencilpt[fill=blue]{1,1}{i+1j+1}{};
			
			\draw (i+1j) -- (ij);
			\draw (i-1j) -- (ij);
			
			\draw [->](ij) -- (ij+1);	
			\draw [->](i+1j+1) -- (ij+1);
			\draw [->](i-1j+1) -- (ij+1);
		\end{tikzpicture}
		\caption{Graphical representation of the Crank-Nicolson scheme dependencies.}
		\label{fig:stencil:crank-nicolson}
	\end{figure}
	Described earlier scheme is first-order accurate both in space and time. This method is unconditionally stable for: $C > 0$.
	