\section{Cost of communication} \label{s:results:cost-of-communication}
	A process that performs a few calculations and then sends a single short message may have the same \gls{computation-to-communication-ratio} as a process that performs millions of calculations and then sends many large messages. For each of discussed schemes \gls{computation-to-communication-ratio} is determined based on \gls{stencil} and analysis described in Section \ref{s:general-approach:analysis-of-the-schemes} and \ref{s:general-approach:devise-a-way-to-parallelize}.
	
	Basing on Figure \ref{fig:communication:explicit-upwind} total size of messages that are send in a single time step is equal to $p-1$, where $p$ is total quantity of processors used in calculations. For single time step there is need to compute $n$ points, so \gls{computation-to-communication-ratio} for this scheme is equal to
	\begin{equation}
		\frac{n}{p-1}
	\end{equation}
		
	Implicit upwind scheme also requires to calculate only $n$ values, because of computed equation \eqref{eq:implicitSubstitution}. Total amount of messages send between processors in each time step is the same as in case of explicit upwind scheme and equal to $p-1$ basing on Figure \ref{fig:visualization:implicit-upwind}. Thus ratio in this case is equal to
	\begin{equation}
		\frac{n}{p-1}
	\end{equation}
	
	For Crank-Nicolson scheme calculations are more complex. As mentioned in Section \ref{s:general-approach:crank-nicolson-parallel-algorithm} applying formulations for new solution requires computation of $n$ points in the grid and $2p$ messages. In next step forward substitution requires to calculate another $n$ values exchanging $2(p-1)$ messages. Backward substitution requires only $p-1$ messages and for $n$ points. The overall \gls{computation-to-communication-ratio} for this scheme is equal to
	\begin{equation}
		\frac{3n}{2p + 2(p-1) + (p-1)} = \frac{3n}{5p-3}
	\end{equation}
	
	Comparing calculated ratios we can observe that cost of communication is the greatest for Crank-Nicolson schema. Ratio between number of calculations and total size of messages send by processors in terms of upwind schemes is the same, but for implicit scheme we will observe a standby time (Figure \ref{fig::implicit-upwind}).