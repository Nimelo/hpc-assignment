\section{Analysis of schemes} \label{s:general-approach:analysis-of-the-schemes}

	This section is devoted to briefly introduce reader about formulations, \gls{stencil}s and stability conditions of used schemes. All the formulas use following dependency (\ref{eq:cfl}), also known as \gls{CFL} condition. In order to connect error of calculated norm 1 with a single point in grid formulation of this norm was changed and is explained in appendix \ref{s:appendix:norms:calcNorm1} by Formula \eqref{eq:calcNorm1}.	
	\begin{equation} \label{eq:cfl}
	C = u \frac{\Delta t}{\Delta x}
	\end{equation} 	
	where $C$ is the \gls{CFL} number. In initial parameters of \gls{advection-equation} \eqref{eq:advection-equation} give us $u > 0$ and $\Delta x$ is also greater than $0$ because it refers to amount of points in the grid, which cannot be negative and solution for zero points doesn't make sense. The last important parameter is $\Delta t$ which also should be greater than zero, because it is impossible to go back in time. The conclusion from Equation \eqref{eq:cfl} is that the \gls{CFL} should always be positive in conditions of considered in this report problem. There is a possibility to change this assumptions but, we will lose the ability to use the following equation $\Delta t = \frac{C\Delta x}{u}$.
	
	For all graphical, geometric arrangement of a nodal group that relate to the point of interest by using a numerical approximation routine in this report used is following notations:
	
	\begin{itemize}
		\item blue nodes are points at time step $n$,
		\item green nodes are points at time step $n+1$,
		\item red nodes are points at time step $n+const$, where $const$ is a number,
		\item directed lines show how relations between points of different time steps,
		\item undirected lines show relations between points of same time step,
		\item if scheme is multi-step there will be number corresponding to the step above directed line.
	\end{itemize}

\section{Explicit upwind scheme} \label{s:general-approach:explicit-upwind-analysis}
Formula that gives first-order accurate method for our problem is described by Klaus A. Hoffman and Steve T. Chiang in \cite{bib:hoffman}[p. 191 - 192] in Section 6.5 titled Applications to a Linear Problem, which looks as follows:
\begin{equation} \label{eq:explicitUpwind}
	\begin{split}
		\frac{f_i^{n+1} - f_i^n}{\Delta t} + u\frac{f_i^n - f_{i-1}^n}{\Delta x} = \mathcal{O}(\Delta t, \Delta x) \\
		f_i^{n+1} = f_i^n - C(f_i^n - f_{i-1}^n) + \mathcal{O}(\Delta t^2, \Delta x \Delta t)
	\end{split}
\end{equation}
where $C$ is \gls{CFL} number introduced earlier \eqref{eq:cfl}. Geometric arrangement of a nodal group that relate to the points of interest used by this scheme is shown in Figure \ref{fig:stencil:explicit-upwind}.
\begin{figure}[!htbp]
	\centering
	\begin{tikzpicture}[scale=1.5]
		\draw[dotted] (-4,0.5) -- (4,0.5);
		\draw[dotted] (-4,-0.5) -- (4,-0.5);
		\node[black] at (-4.5,1) {$n+1$};
		\node[black] at (-4.5,0) {$n$};
		
		\node[black] at (-1,-1) {$i-1$};
		\node[black] at (0,-1) {$i$};
		\node[black] at (1,-1) {$i+1$};
		
		\stencilpt[fill=blue]{0,0}{ij}{};
		\stencilpt[fill=blue]{-1,0}{i-1j}{};			
		\stencilpt[dotted]{1,0}{i+1j}{};
		
		\stencilpt[fill=green]{0,1}{ij+1}{};
		\stencilpt[dotted]{-1,1}{i-1j+1}{};			
		\stencilpt[dotted]{1,1}{i+1j+1}{};
		\draw (ij) -- (i-1j);
		\draw [->](i-1j) -- (ij+1);
		\draw [->](ij) -- (ij+1);	
	\end{tikzpicture}
	\caption{Graphical representation of the explicit upwind scheme dependencies.}
	\label{fig:stencil:explicit-upwind}
\end{figure}
Described earlier scheme is first-order accurate both in space and time. This method is conditionally stable for: $C \in (0,1)$.

\section{Implicit upwind scheme} \label{s:general-approach:implicit-upwind-analysis}
	This formulation uses the \gls{FTBS} method of finite differencing in the approximation of the \gls{PDE} in \eqref{eq:advection-equation}, therefore giving:
	\begin{equation} \label{eq:implicitUpwind_first}
		\frac{f_i^{n+1} - f_i^n}{\Delta t} + u\frac{f_i^{n+1} - f_{i-1}^{n+1}}{\Delta x} = \mathcal{O}(\Delta t, \Delta x)
	\end{equation}
	The only known value of $f$ at $i$ is at time $n$. Mentioned dependency drives us to solve following linear equation set of form $Af^{n+1} = f^n$:		
	\begin{equation}
		\begin{bmatrix}
			1+C & & & \\
			-C & 1+C & & \\ 
			& \ddots & \ddots \\
			& & -C & 1+C \\					
		\end{bmatrix} 
		\times
		\begin{bmatrix}
			f_1^{n+1} \\
			f_2^{n+1} \\
			\vdots	\\
			f_{N-1}^{n+1}\\
		\end{bmatrix}
		=
		\begin{bmatrix}
			f_1^{n} + C f_0^{n+1}\\
			f_2^{n} \\
			\vdots	\\
			f_{N-1}^{n}\\
		\end{bmatrix}
	\end{equation} 			
	where $N$ is number of grid points.
	This method is first--order accurate both in time and space. We can observe that we do not have to solve it by using complicated algorithm. Matrix $A$ is bidiagonal diagonally-dominant, so although this schema is implicit we can solve it using forward or backward substitution described using formula \eqref{eq:implicitSubstitution}. This equation assumes that we know the boundary conditions for next time step -- $f_0^{n+1}$ and $f_N^{n+1}$.  Graphical interpretation of this method is shown in Figure \ref{fig:stencil:implicit-upwind}.
	\begin{figure}[!htbp]
		\centering
		\begin{tikzpicture}[scale=1.5]
			\draw[dotted] (-4,0.5) -- (4,0.5);
			\draw[dotted] (-4,-0.5) -- (4,-0.5);
			\draw[dotted] (0.5, 2) -- (0.5, -1.5);
			
			\node[black] at (-1, 1.75) {$C>0$};
			\node[black] at (2, 1.75) {$C<0$};
			\node[black] at (-4.5,1) {$n+1$};
			\node[black] at (-4.5,0) {$n$};
			
			\node[black] at (-2,-1) {$i-1$};
			\node[black] at (-1,-1) {$i$};
			\node[black] at (0,-1) {$i+1$};
			
			\node[black] at (1,-1) {$i-1$};
			\node[black] at (2,-1) {$i$};
			\node[black] at (3,-1) {$i+1$};
			
			\stencilpt[dotted]{-2,0}{i-2j}{};
			\stencilpt[fill=blue]{-1,0}{i-1j}{};
			\stencilpt[dotted]{0,0}{ij}{};
			\stencilpt[dotted]{1,0}{i+1j}{};	
			\stencilpt[fill=blue]{2,0}{i+2j}{};
			\stencilpt[dotted]{3,0}{i+3j}{};
			
			\stencilpt[fill=blue]{-2,1}{i-2j+1}{};
			\stencilpt[fill=green]{-1,1}{i-1j+1}{};
			\stencilpt[dotted]{0,1}{ij+1}{};
			\stencilpt[fill=green]{1,1}{i+1j+1}{};
			\stencilpt[fill=blue]{2,1}{i+2j+1}{};
			\stencilpt[dotted]{3,1}{i+3j+1}{};
			
			\draw (i-2j+1) -- (i-1j);
			\draw [->](i-2j+1) -- (i-1j+1);
			\draw [->](i-1j) -- (i-1j+1);
			
			\draw (i+2j) -- (i+2j+1);
			\draw [->](i+2j+1) -- (i+1j+1);
			\draw [->](i+2j) -- (i+1j+1);
		\end{tikzpicture}
		\caption{Graphical representation of the implicit upwind scheme dependencies.}
		\label{fig:stencil:implicit-upwind}
	\end{figure}
	\begin{equation} \label{eq:implicitSubstitution}
		\begin{split}
			f_i^{n+1} = \frac{f_i^n + Cf_{i-1}^{n+1}}{1+C}\text{, for $i = 1,2,3,\ldots,N-1$ and $C>0$} \\
			f_{i-1}^{n+1} = \frac{(1+C)f_i^{n+1} - f_i^{n+1}}{C}\text{, for $i = N, N-1, \ldots, 2$ and $C < 0$}
		\end{split}
	\end{equation}
	This schema is stable for $C \in (-\infty, -1) \cup (0, \infty)$, so in terms of assignment ($C>0$) scheme is unconditionally stable. 
\subsection{Crank-Nicolson scheme} \label{s:general-approach:crank-nicolson-analysis}
	Based on formulations given by S. Scott Collis in \cite{bib:introduction} and George Em Karniadakis and Robert M. Kirby II in \cite{bib:mpi} for Crank--Nicolson scheme to solve \gls{advection-equation} can be described as:
	\begin{equation} \label{eq:crank-nicolson}
		\frac{f_i^{n+1} - f_i^n}{\Delta t} +  u \frac{1}{2} \bigg(\frac{f_{i+1}^{n+1} - f_{i-1}^{n+1}}{2\Delta x} + \frac{f_{i+1}^{n} - f_{i-1}^n}{2\Delta x}\bigg)= \mathcal{O}(\Delta t, \Delta x)
	\end{equation}
	Equation \eqref{eq:crank-nicolson} can be transformed in a following fashion to separate knowns and unknowns:
	\begin{equation} \label{eq:crank-nicolson:solution}
		\frac{C}{4} f_{i-1}^n + f_{i}^n - \frac{C}{4} f_{i+1}^n = -\frac{C}{4} f_{i-1}^{n+1} + f_{i}^{n+1} + \frac{C}{4} f_{i+1}^{n+1}
	\end{equation}
	where $C$ is \gls{CFL} number introduced earlier \eqref{eq:cfl}. Assuming following dependencies:
	\begin{equation}
		\begin{split}
			\alpha &= \frac{C}{4}\\
			 k_i &= \frac{C}{4} f_{i-1}^n + f_{i}^n - \frac{C}{4} f_{i+1}^n
		\end{split}
	\end{equation}
	equation \ref{eq:crank-nicolson:solution} drives us to solve following linear system of form $Af^{n+1} = f^n$ by Thomas algorithm:		
	\begin{equation}
		\begin{bmatrix}
			1 		& \alpha &		  &  	 &		&\\
			-\alpha & 1 	 & \alpha &  	 &		&\\ 
					& \ddots & \ddots &  	 &		&\\
					&		 &-\alpha & 1 	 &\alpha&\\
					& 		 & 		  &-\alpha&	1	&\\					
		\end{bmatrix} 
		\times
		\begin{bmatrix}
			f_1^{n+1} \\
			f_2^{n+1} \\
			\vdots	\\
			f_{N-1}^{n+1}\\
			f_{N}^{n+1}\\
		\end{bmatrix}
		=
		\begin{bmatrix}
			k_0 \\
			k_1 \\
			\vdots	\\
			k_{N-1} \\
			k_{N} \\
		\end{bmatrix}
	\end{equation}		
	Geometric arrangement of a nodal group that relate to the points of interest used by this scheme is shown in Figure \ref{fig:stencil:crank-nicolson}.
	\begin{figure}[!htbp]
		\centering
		\begin{tikzpicture}[scale=1.5]
			\draw[dotted] (-4,0.5) -- (4,0.5);
			\draw[dotted] (-4,-0.5) -- (4,-0.5);
			\node[black] at (-4.5,1) {$n+1$};
			\node[black] at (-4.5,0) {$n$};
			
			\node[black] at (-1,-1) {$i-1$};
			\node[black] at (0,-1) {$i$};
			\node[black] at (1,-1) {$i+1$};
			
			\stencilpt[fill=blue]{0,0}{ij}{};
			\stencilpt[fill=blue]{-1,0}{i-1j}{};			
			\stencilpt[fill=blue]{1,0}{i+1j}{};
			
			\stencilpt[fill=green]{0,1}{ij+1}{};
			\stencilpt[fill=blue]{-1,1}{i-1j+1}{};			
			\stencilpt[fill=blue]{1,1}{i+1j+1}{};
			
			\draw (i+1j) -- (ij);
			\draw (i-1j) -- (ij);
			
			\draw [->](ij) -- (ij+1);	
			\draw [->](i+1j+1) -- (ij+1);
			\draw [->](i-1j+1) -- (ij+1);
		\end{tikzpicture}
		\caption{Graphical representation of the explicit upwind scheme dependencies.}
		\label{fig:stencil:crank-nicolson}
	\end{figure}
	Described earlier scheme is first-order accurate both in space and time. This method is unconditionally stable for: $C > 0$.
	