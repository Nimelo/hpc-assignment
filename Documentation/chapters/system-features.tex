\chapter{System Features} \label{chp:system-features}
	\begin{comment}
		$<$This template illustrates organizing the functional requirements for the 
		product by system features, the major services provided by the product. You may 
		prefer to organize this section by use case, mode of operation, user class, 
		object class, functional hierarchy, or combinations of these, whatever makes the 
		most logical sense for your product.$>$
	\end{comment}
	This chapter is divided into four main sections, which are responsible to contain list of requirements and explain processes within among those requirements. First section \ref{s:system-features:communication-feature} describes all the requirements connected with the user-interface, such us the responses for given action parameters. Next section \ref{s:system-features:configuration-feature} presents the way of configuring the system, containing very specified requirements. Simulation process is described in a \ref{s:system-features:simulation-feature} section. The last section \ref{s:system-features:reporting-feature} standardize the report format and the calculated values.
	
	\input{communication-feature}
	\input{configuration-feature}	
	\input{simulation-feature}
	\input{raporting-feature}