\chapter{Introduction} \label{chp:introduction}
	The scientific computing is very popular and emphasized across almost all fields of science where calculations are made. If there is a possibility to speedup computations without acquiring new better and faster machine it is highly desired that such prospect will be utilized. Nowadays almost all processors have multiple amount of cores, that can work simultaneously. By today's standards everyone how has a decent machine can have access to own \gls{super-computer}.
	
	Imagine a situation where scattered job can be two or more times efficient in terms of execution, it gives a huge boost especially for very long and complicated problems. Many of real-world issues does not make sense if results of calculations exceed the deadline, e.g. weather forecasting. For mentioned before meteorology problems some simplifications can be made to even more speed up the process, but there are many problems that cannot be simplified or speedup using those techniques. The only way to improve the performance of calculations is to parallelize the algorithm. If it is possible than gain of the speedup will be noticed immediately as long as time of communication is significantly lower than time of parallel computations.	 
	
	Engineers from all around the world struggle with the execution time of their algorithms. The faster algorithm is the better it is, nowadays nobody take care about memory complexity, because it's cheaper with each year. There are a lot of tools for simulation even using parallel approach, but sometimes there is a need to implement something custom. In case of this document numerical schemes should be implemented using \emph{C++} language with \gls{MPI} for university \gls{super-computer}. It requires knowledge of numerical schemes as well as experience in programming.
	
	Section \ref{s:introduction:objective} concentrates on the objective of assignment. Introduces various of tasks to be completed in following chapters. Chapter \ref{chp:general-approach} describes used numerical methods and a ways to parallelize mentioned schemes. Next Chapter \ref{chp:results} is divided into three sections. First one \ref{s:results:compare-solutions-serial-parallel} compares numerical solutions to the analytical ones across serial and parallel algorithms. Next section \ref{s:results:performance-serial-parallel} presents theoretical and practical performance of serial and parallel algorithms. Last Section \ref{s:results:performance-external-linear-solver} will show the impact of using different implementation of linear equation solver for serial algorithm. Chapter \ref{chp:summary} contains all the conclusions and comments about \gls{HPC}, \gls{MPI} and process of developing and implementing numerical schemes for \gls{advection-equation} for \gls{super-computer}s. 
	
	
	\section{Objective} \label{s:introduction:objective}
	The main objective of the assignment is apply distributed memory parallel programing techniques, implement, examine and discuss about three numerical schemes (\emph{implicit upwind}, \emph{explicit upwind} and \emph{Crank-Nicolson}) used to solve simple \gls{advection-equation} which is described as follows:	
	\begin{equation} \label{eq:advection-equation}
	\frac{\partial f}{\partial t} +  u\frac{\partial f}{\partial x} = 0
	\end{equation}
	in a domain $x \in [-50,50]$ with $u = 1.75$ and following initial/boundary conditions:	
	\begin{equation} \label{eq:boundary-conditions}
		\begin{split}
			f(x, 0) &= \frac{1}{2} exp(-x^2) \\
			f(-50, t) &= 0 \\
			f(50, t) &= 0
		\end{split}
	\end{equation}
	The analytical solutions for this initial conditions are given by:
	\begin{equation} \label{eq:analytical-solution}
			\begin{split}
				f(x, t) = \frac{1}{2} exp \bigg( -(x - 1.75t)^2 \bigg).
			\end{split}
	\end{equation}
	
	Solution should be implemented in \emph{C/C++/FORTRAN} and coupled with \gls{MPI}. Trials should be performed on university \gls{super-computer} -- \gls{astral}. A way to parallelize the above numerical methods should be described. Comparison of numerical solutions obtained for both serial and parallel programs should be made with analytical solution. Cost of the communication in the boundary exchange and the cost of computing a time-step for each process all should be included as well as performance of serial and parallel code in terms of practical trials. In the end replacement of linear equation set should be made in order to check the how good is current implementation.
	