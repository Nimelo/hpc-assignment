\begin{abstract}
	Scientific computing is very important and handy discipline of science. It uses advanced computing capabilities to solve complex problems. Having very effective and fast algorithm is very crucial in terms of execution time. Also important is to combine those algorithm with current technology and make use of a network of supercomputers to obtain the best possible performance.
	
	Parallel computing which is a very desire type of computation in which many, not always all, calculations are carried out simultaneously or in parallel. There are different levels of parallelism. In this report only small part will be discussed. Those calculations are used in many fields of science. The \gls{HPC} calculations are used in weather forecasting, aerodynamic research, probabilistic analysis, brute force code breaking, molecular dynamics simulations and many different domains.
	
	This report summarizes work and explains approaches and techniques of developing numerical schemes for \gls{super-computer} using \gls{MPI}. Numerical schemes will solve simple \gls{PDE} in certain domain and boundary conditions. The mentioned solutions for \gls{advection-equation} are acquired by three different numerical schemes: \emph{explicit upwind}, \emph{implicit upwind} and \emph{Crank-Nicolson} schemes.
	
	Reflection of a parallelization problem is made by a devision of a way to parallelize above numerical methods. Complexities of the parallel algorithms were compared between themselves as well as their real time execution and speedup relative to their serial equivalent. Theoretical cost of communication and calculation is also described and compared to empirical trials. At the end of a report there are comments regarding to \gls{MPI} parallelization problem for discussed numerical methods.
\end{abstract}